\documentclass[10pt]{article}

\usepackage[left=1in,right=1in,top=1in,bottom=1in]{geometry}
\usepackage{amssymb}
\usepackage{enumitem}
\usepackage{graphicx}
\usepackage{array}
\newcolumntype{L}{>{\centering\arraybackslash\small}m{3cm}}
\newcolumntype{K}{>{\centering\arraybackslash\small}m{10cm}}

\begin{document}

\title{CS 4341: Homework 1}
\author{Adam Camilli (aocamilli@wpi.edu)}
\date{\today}
\maketitle

\section*{Ch 1: Introduction}
\begin{enumerate}
\item \textbf{Problem 1.7:} To what extent are the following computer systems
  instances of artificial intelligence?
  \begin{itemize}
  \item Supermarket bar code scanners. \\~\\ 
     Upon sensing a barcode provided by a buyer or cashier they
    simply look up the barcode in a linked database. They are a good example
    of electrical engineering backed up by a very elementary DB-lookup
    program. The process of
    reading a barcode is not akin whatsoever to human sight: The bars
    are simply a way of representing numbers that can be read by a
    photoelectric cell that converts the zebra pattern into on-off
    pulses representing binary numbers. Nothing about this process is
    autonomous, let alone intelligent. Therefore, supermarket bar code
    scanners are not a good example of AI.
  \item Web search engines. \\~\\ 
    Web search engines have a number of features that can be
    interpreted as intelligent, such as use of data mining to detect
    typing errors. Every modern search engine makes
    use of machine learning and natural language
    processing as well. However, it is hard to argue that search engines themselves
    are rational agents: Their ability to ``learn'' is extremely
    specialized and not at all extendable to other classes of
    problems. Search engine design is mainly based in development of algorithms
    such as PageRank in combination with efficient DB design and
    processing power. Search
    engines do utilize many techniques from AI development and so are a much better example of AI than bar code
    scanners. But they are not autonomous and are only intelligent in
    certain very specific instances. 
  \item Voice-activated telephone menus. \\~\\
    Obviously these devices are a fantastic example of the power of
    natural language processing, considered one of the
    hardest features to implement in AI. But these devices are not
    really intelligent whatsoever: Like search engines, they are able
    to analyze certain  kinds of patterns in a way that is
    similar to human, but this ability is rooted in very specialized
    engineering that is not extendable to general pattern
    recognition. Some speech recognition systems can become attuned to
    a certain speaker's voice, but telephone voice systems are generally
    speaker-independent devices which cannot do so. As experience with
    Siri shows, however, deep learning has enabled these devices are to interact
    with humans in a dynamic and improvised manner. This represents a
    much bigger advancement than barcode scanners which simply rely on
    lookup tables, and the intelligence required can be argued to act
    rationally in a manner search engines cannot: Siri needs to string
    together sentences, a task far harder for computers than
    displaying a list. Voice-activated systems are therefore an even
    better example of AI than search engines, though they are
    definitely not a
    complete rational agent. 
  \item Internet routing algorithms that respond dynamically to the state of their network. \\~\\ 
    Although these algorithms make little to no use of natural
    language processing, they are required to act rationally in terms
    of cost-benefit analysis, to a much greater extent than search
    engines or voice processors. Routers need to accomplish several
    goals at once: The goal of being fast, the goal of transmitting
    complete and non-corrupted data, not consuming an excess amount of
    power, etc. The algorithms to dynamically work to accomplish these
    tasks are great examples of how a device acts rationally. With
    constantly varying input (network status), they are able to adjust
    how their output (packets) is delivered. Out of all the devices,
    it can be argued this is the most complete example of a rational
    agent, and therefore of AI.
  \end{itemize}
\item Do you think that the Turing test is a good way to judge a rational agent? Justify
  your answer. \\ \\
  The Turing test is useful as a way to understand the metrics of how a machine can pass as human. But it is far too abstract and difficult to determine rationality, which a machine can easily possess while still being machinelike.
  
  
\end{enumerate}

\newpage

\section*{Ch 2: Agents}
\begin{enumerate}
\item Problem 2.4 \\ \\
    \begin{tabular}{|L||L|L|L|L|}
      \hline  
      \textbf{Agent Type} & \textbf{Performance Measure} & \textbf{Environment} & \textbf{Actuators} & \textbf{Sensors} \\
      \hline
     Robotic Agent: Soccer player & Win-Loss record, goals for-against, possession & Field, ball, own/other team, own body, teammate and opponent body & Legs for moving, feet/head/body for manipulating ball, arms/appendages for balance/fighting opponents & Camera, touch and orientation sensors, speedometer, accelerometer \\
      \hline
      Robotic Agent: Titan subsurface ocean explorer & Data retrieved
                                                       (pictures and
                                                       video) & Ice
                                                                desert,
                                                                water
                                                                lakes
                                                                under
                                                                ice &
                                                                      Legs for traversing surface, drill to break through ice, propellers for moving, rudders+fins for steering & Camera, thermometer, mass spectrometer, chemical sensor (to detect non-water lakes), orientation sensor, speedometer, radio/satellite receiver \\
      \hline
      Software Agent: Used AI book shopper & Obtaining used books pertaining to AI from online markets & Internet & Follow link, submit queries, enter data in fields & HTML parser, scam detector, algorithm to use website data to determine if products are used and about AI \\
      \hline
      Robotic Agent: playing tennis against an opponent & Win-Loss record, not making unforced errors & Turf and clay courts, ball, racket, own body, opponent body & Legs for moving, feet for balancing, gripping appendages & Camera, separate ball and opponent spacial trackers, orientation sensor, joint controllers for switching between forehand and backhand \\
      \hline
      Robotic Agent: playing tennis against a wall & Repeatedly hitting and returning ball & Half-sized court with wall instead of net, own body, ball, racket & Legs, feet, gripping appendage & Camera, ball tracker, orientation sensor, joint controllers \\
      \hline
      Robotic Agent: Performing a high jump & Jumping over as high a bar as possible without touching it & Straight track, bar, mat & Legs, optimized leg joint controllers to jump as high as possible, additional joint controllers to elongate body during jump and twist body before jump & Camera, orientation sensor, own center of gravity detector, distance tracker to bar \\
      \hline
      Robotic Agent: Knitting a sweater & Knitting a wearable sweater quickly and efficiently & Pile of yarn on spool, knitting needles, frame to hold fabric and spin yarn & Appendages with digits to manipulate yarn and vary stitches, needle operating device + motor, motor to spin yarn & Camera, pattern recognition + scheduler \\
      \hline
      \newpage
      Software Agent: Bidding on item at an auction & Outbid others (obtain items) as economically as possible & Item list, auctioneer, own bid, other bids & Bid signaler (hand, data submitter, etc.), exit function when done & Auction start sensor, bids vs real-value tracker, current highest-bid tracker \\
      \hline                                                                 
    \end{tabular}

\newpage

\item Problem 2.5 \\ \\
  \begin{tabular}{|L|K|}
    \hline
    \textbf{Term} & \textbf{My definition} \\
    \hline
    Agent & An autonomous entity that observes its environment with sensors, and then intelligently acts upon the environment with actuators. \\
    \hline
    Agent Function & A function that mathematically relates an agent's percept sequence to an action \\
    \hline
    Agent Program & A computer program that takes input from an agent's sensors and outputs an action to its actuators \\
    \hline
    Rationality & The concept of maximizing the usefulness of one's actions given one's current knowledge \\
    \hline
    Autonomy & The ability of an agent to act for itself, independent of the influence of any other agents or entities \\
    \hline
    Reflex Agent & An agent that only acts in response to a direct stimulus, using only its immediate perception of the stimulus to make a decision. \\
    \hline
    Model-Based Agent & An agent that possesses knowledge of its environment in the form of a mathematical or scientific model, and can use it to reach decisions. \\
    \hline
    Goal-Based Agent & An agent that possesses knowledge of its environment and of a goal it desires to reach. It collects multiple possible actions using perception and a model, and selects the action that matches its goal. \\
    \hline
    Utility-Based Agent & An agent with knowledge of the world and its goal, and an internal utility function to quantify how useful an action is in terms of reaching its goal. It then chooses the action that matches its goal and has the highest utility. \\
    \hline
    Learning Agent & An agent with the ability to internalize what it perceives and become more effective as it gains more knowledge of its environment. It has a learning element to make improvements, and a performance element to select actions. The learning element ideally increasingly optimizes the performance element as knowledge is gained. \\
    \hline
  \end{tabular}
\end{enumerate}

\section*{Ch 3: Search}
\begin{enumerate}
\item Problem 3.2 
  \begin{enumerate}
  \item
    \begin{itemize}
    \item States: A graph whose nodes can point north, south, east and
      west, or to a sentinel node when adjoining a wall; Current orientation; Current node
    \item Initial state: Graph with one central node, pointing north.
    \item Actions: Turn north, south, east, west; Move forward
    \item Transition Model: Move in current direction if not facing wall, if facing wall stop.
    \item Goal Test: Is most recent node out of the maze?
    \item Path Cost: None
    \end{itemize}  
State space is infinite.
  \item
    \begin{itemize}
    \item States: A graph whose nodes can point north, south, east and
      west, and which indicate if there is a neighbor in that direction; Current orientation; Current node
    \item Initial state: Graph with one central node, pointing north.
    \item Actions: Turn north, south, east, west if at an intersection; Move forward
    \item Transition Model: Move in current direction if not facing wall or at an intersection, if facing wall stop, else if at an intersection turn clockwise (north$\rightarrow$ east$\rightarrow$ south 
$\rightarrow$ west)
    \item Goal Test: Is most recent node out of the maze?
    \item Path Cost: None
    \end{itemize}
    State space is infinite.
  \item
    \begin{itemize}
    \item States: A graph whose nodes can point north, south, east and
      west, and which indicate if it is a turning point; Current orientation; Current node
    \item Initial state: Graph with one central node, pointing north.
    \item Actions: Turn north, south, east, west if at a turning point; Move forward
    \item Transition Model: Move in current direction if not facing wall or at an intersection, if facing wall stop, else if at an intersection turn clockwise (north$\rightarrow$ east$\rightarrow$ south 
      $\rightarrow$ west)
    \item Goal Test: Is most recent node out of the maze?
    \item Path Cost: None
    \end{itemize}  
    State space is infinite, current orientation is now irrelevant.
  \item
    \begin{enumerate}
    \item The passages are all straight and passable at equal costs
    \item The robot has unlimited energy
    \item The maze is oriented perfectly north, south, east and west
    \end{enumerate}
  \end{enumerate}
\item Problem 3.6 
 \begin{enumerate}
  \item 4-Color
    \begin{itemize}
    \item States: A graph where nodes correspond to regions and edges are drawn between nodes representing adjacent regions; Current node; Node queue.
    \item Initial state: Planar graph with uncolored nodes.
    \item Actions: Color node; Add child(ren) to queue
    \item Transition Model: Add all nodes to queue. Color current node, remove it from queue, move to next node in queue.
    \item Goal Test: Graph is colored (no more nodes in queue) and no adjacent nodes have same color.
    \item Path Cost: None
    \item State space is bounded at $4^n$ where $n$ is number of regions (nodes).
    \end{itemize}  
  \item Monkeys.
    \begin{itemize}
    \item States: Square room with $n*n$ tiles, $n^2$ possible banana
      arrangements; Current node (monkey); Has bananas; Has crates;
      Current height.
    \item Initial state: 
    \item Actions: Move up, down, left, right; Move crates up, down,
      left, right; Climb crates; Grab bananas.
    \item Transition Model: Move up, down, left, right. If nothing
      found move random direction. If wall found, turn around. If
      crate found, pick it up. If two crates found, plant and
      climb. If no bananas found, climb down, pick up crates, move
      random direction and repeat.
    \item Goal Test: Monkey has the bananas.
    \item Path Cost: None
    \item Uses breadth-first search of room.
    \end{itemize}
  \item Input records.
    \begin{itemize}
    \item States: An array of records, one or more illegal; Current
      index; List of indices of illegal records.
    \item Initial state: Array of records, current index 0, empty illegal list.
    \item Actions: Inspect record, move to next record.
    \item Transition Model: If illegal, add index to illegal
      list. Else do nothing. Repeat for whole array.
    \item Goal Test: Inspect whole array.
    \item Path Cost: None (illegal or not, assumed negligible)
    \end{itemize}  
  \item Jugs.
    \begin{itemize}
    \item States: Three jugs with different states of fullness.
    \item Initial state: All three jugs empty.
    \item Actions: Fill desired jug, empty into desired jug, empty
      desired jug on ground
    \item Transition Model: Jug can be completely filled, completely
      emptied, or increased/decreased by size of another jug.
    \item Goal test: Does any jug contain exactly one gallon?
    \end{itemize}
  \end{enumerate}
\item Problem 3.10 \\ \\
  \begin{tabular}{|L|K|}
    \hline
    \textbf{Term} & \textbf{My definition} \\
    \hline
    State & A collection of parameters and information representing an agent's
            perception of its environment at a particular moment in time. \\
    \hline
    State Space & The set of all possible states in a given system. \\
    \hline
    Search Tree & A tree data structure used to hold keys of a set so they can be traversed or searched for by an agent in an algorithmic fashion. \\
    \hline
    Search Node & A key in a search tree connected to a parent key (unless it is the root node) and to 0, 1, or many child keys. \\
    \hline
    Goal & A state that is desirable to an agent. \\
    \hline
    Action & A change from one state to another by an agent. \\
    \hline
    Transition Model & A mathematical or logical model describing the results of a set of actions. \\
    \hline
    Branching Factor & The average number of children of each node in a search tree. \\
    \hline
  \end{tabular}
\item Problem 3.14 
  \begin{enumerate}
    \item False: Depth-first search may sometimes directly find goal with no backtracking, which outperforms A*.
    \item True: $h(n)$ never over-estimates remaining optimal distance to goal node.
    \item False: A* still works in continuous situations as a heuristic.
    \item True if branching factor is finite. If this is the case any goal node at a finite depth will be found.
    \item False: Rooks can move many spaces in a straight line, and so Manhattan is not admissible.
  \end{enumerate}
\item Problem 3.21 
  \begin{enumerate}
  \item True: If step costs are equal, $g(n) \propto$ depth.
  \item True: Depth-first search is Best-first search with $f(n) = depth(n)$.
  \item True: Uniform-cost search is A* with $h(n) = 0$.
  \end{enumerate}
\item In this problem, you are asked
  to select the best search strategy for each of the situations given
  below. You can choose any of the informed or uninformed search
  methods discussed in class. Justify your answer briefly but
  decisively.  
  \begin{enumerate}
  \item All the arc (= action) costs are equal. No heuristic
  information is given. Time is not a problem, and you have plenty of
  space. But you must find an optimal solution. \\ \\
  Since we have no heuristic and time is not an issue, we need to select an uninformed search strategy that is complete and finds an optimal solution. Due to its ease of implementation, I suggest bread-first search.
  \item Cost and heuristic
  information are given. You need to find an optimal solution and you
  want to save as much time as possible.  \\ \\
  In this case, choose A*: It is optimal and complete, and most of the time is the most efficient heuristic search.
  \item c) The search space is large
  and the depth of the solution is not known. No cost and no heuristic
  information are given. You need to find the shallowest solution but
  you don't have much memory/space available. \\ \\
  Here we need an uninformed search strategy that only goes as deep as it needs to, but is also complete and optimal. Therefore I suggest iterative deepening search.
  \end{enumerate}

\item Problem 11 attached separately.

\item Problem 5.9 \\ \\
  \begin{enumerate}
  \item Upper bound is $9! = 362880$, true figure (since games end 5, 6, 7, 8 moves) is 255168.
  \item to e. attached separately.
  \end{enumerate}

\item Problem 13 attached separately.

\end{enumerate}

\end{document}