\documentclass[12pt]{article}

% Popup for cipher
\usepackage{pdfcomment}
% Draw Caesar cipher / LFSR
\usepackage{tikz}
\usetikzlibrary{matrix,positioning,calc,fit}
\tikzset{circ/.style={draw,circle,node distance=2mm,inner sep=0.1pt},topath/.style={to path={|-(\tikztotarget)}}}

\usepackage{ifthen}
% Be able to get substrings
\usepackage{xstring}

% Caesar cipher drawing
\newcounter{index}
%command to increase a counter by 1 modulo 26
\newcommand{\increase}[1]{\ifthenelse{\arabic{#1}<26}{\addtocounter{#1}{1}}{\setcounter{#1}{1}}}
\newcommand{\pOneCipher}{CBDGFAIELKHJMSNQOPURVTYWXZ}
\newcommand{\pOneKey}{ETANORSICLHDUPMGWFVYKBXJQZ}

% Standard
\usepackage[top=0.8in,bottom=0.8in]{geometry}
\usepackage{amsmath,bm}
\usepackage{amssymb}
\usepackage{physics}
\usepackage{enumitem}
\usepackage{graphicx}
\usepackage{tabularx}
\usepackage{tcolorbox}
\usepackage{array,multirow}
\usepackage{mathtools}
\DeclarePairedDelimiter{\ceil}{\lceil}{\rceil}
\DeclarePairedDelimiter\floor{\lfloor}{\rfloor}
% Enables font change size in verbatim blocks
\makeatletter
\newcommand{\verbatimfont}[1]{\renewcommand{\verbatim@font}{\ttfamily#1}}
\makeatother
\usepackage{hyperref}
\hypersetup{
    colorlinks=true,
    linkcolor=blue,
    filecolor=magenta,      
    urlcolor=cyan,
}
 
\urlstyle{same}


% Format for answering questions
\newenvironment{answer}
{ \begin{tcolorbox}[halign=left]
    }
    {  
  \end{tcolorbox}
}

% Math table column
\newcolumntype{L}{>{$}l<{$}} % math-mode version of "l" column type

\begin{document}

\title{CS 4801: Assignment 4}
\author{Adam Camilli (aocamilli@wpi.edu)}
\date{\today}
\maketitle

\begin{enumerate}

\item Consider the multiplicative group of $\mathbb{Z}^{*}_{79}$.
  \begin{enumerate}
  \item What are the possible element order? How many element exist
    for each order?
    \begin{answer}
      \[ \mathbb{Z}^{*}_{79} = a \in \mathbb{Z}_{79} \mid a \perp 79\]
      Therefore we may count the size of $\mathbb{Z}^{*}_{79}$ with Euler's totient function (for prime numbers, since 79 is prime):
      \[ \phi(79^1) = 79^{1-1}(79-1) = 78\]
      There are 78 elements in $\mathbb{Z}^{*}_{79}$ from $\mathbb{Z}_{79} = \{0,1,\ldots 78\}$:
      \[ \mathbb{Z}^{*}_{79} = \{1,2,\ldots,78\} \]
      The order $m$ of the set $\mathbb{Z}^{*}_{79}$ is just the count of elements 78.
      Therefore the possible element orders are the elements which divide 78:
      \begin{align*}
        78 &= 2^1 \cdot 3^1 \cdot 13^1 \\
           &\Downarrow \\
        78 &= 6 \cdot 13 \\
           &= 2 \cdot 39 \\
           &= 3 \cdot 26 \\     
           &= 1 \cdot 78
      \end{align*}
      This gives us the possible element orders. To find how many elements exist for each, we use Euler's totient function:\\~\\
      \centering\begin{tabular}{|L|L|L|L|}
                  \hline     
                  \text{Order} & \text{\# Elements} & \text{Order} & \text{\# Elements} \\
                  \hline
                  6 & \phi(6) = 2 & 13 & \phi(13) = 12\\
                  \hline
                  2 & \phi(2) = 1 & 39 & \phi(39) = 24\\
                  \hline
                  3 & \phi(3) = 2 & 26 & \phi(26) = 12\\
                  \hline
                  1 & \phi(1) = 1 & 78 & \phi(78) = 24\\
                  \hline
                \end{tabular}                
      
    \end{answer}
  \item Determine the order of all elements of
    $\mathbb{Z}^*_{79}$.  
    \begin{answer}
      To determine the order, we need to use arithmetic modulo 79. The technical definition of an element $a$'s order $\abs{a}$ is the smallest $m$ such that
      \[ a^m = e \]
      where $e$ is the identity element 1. \\~\\ Essentially, to calculate $\abs{a}$, we examine the powers of $a$ to find all values of $m$ where $a^m \bmod 79 = 1$ and $m$ is one of the possible element orders we calculated earlier. \\~\\ There must be at least one such $m$ for all $a \in \mathbb{Z}^*_{79}$, since finite groups are closed under multiplication. In fact, we already know how many $a$'s will fit each possible $m$ since this is found by $\phi(m)$. We can thus simply pick $\phi(m)$ number of $a$'s from the integer solutions of $a^m \bmod 79 = 1$ for each $m$ using one  rule:
      \begin{enumerate}
      \item Exclude results shared with smaller, non-coprime $m$s (i.e. for 39, exclude results shared with 3,6,13, and 26)
      \end{enumerate}
      
      \centering
      \begin{tabularx}{\textwidth}{| X | X | X |}
        \hline     
        Order ($m$) & \# Elements $(\phi(m))$ & Elements $(a^m \bmod 79 = 1)$ of order $m$ \\    
        \hline
        1 & 1 & 1 \\    
        \hline
        2 & 1 & 78 \\    
        \hline
        3 & 2 & $23,55$ \\    
        \hline
        6 & 2 & $24,56$ \\    
        \hline
        13 & 12 & $8,10,18,21,22,$ \newline
        $38,46,52,62,64,$ \newline
        $65,67$ \\   
        \hline
        26 & 12 & $12,14,15,17,27,$ \newline
        $33,41,57,58,61,$ \newline
        $69,71$ \\
        \hline
        39 & 24 & $2,4,5,9,11,$ \newline
        $13,16,19,20,25,$ \newline
        $26,31,32,36,40,$ \newline 
        $42,44,45,49,50,$ \newline
        $51,72,73,76$ \\
        \hline
        78 & 24 & $3, 6, 7, 28, 29,$ \newline
        $30, 34, 35, 37, 39,$ \newline
        $43, 47, 48, 53, 54,$ \newline
        $59, 60, 63, 66, 68,$ \newline
        $70, 74, 75, 77$ \\
        \hline
      \end{tabularx}
    \end{answer}
\newpage
  \item What are the generators of
    $\mathbb{Z}^*_{79}$? 
    \begin{answer}
      For every positive integer $n$, 
      The possible orders are the positive numbers that divide $\texttt{ord}(\mathbb{Z}^*_{79}) = 78$. The generators conversely are those that are coprime to 78:
      \begin{align*}
        78 &= 2^1 \cdot 3^1 \cdot 13^1\\
        \phi(78) &= 78 \prod_{p\mid 78} \left(1-\frac{1}{p}\right) \\
        \phi(78) &= 78 \left(1-\frac{1}{2}\right) \left(1-\frac{1}{3}\right) \left(1-\frac{1}{13}\right) = 24 \\
      \end{align*}
      
      These 24 generators are simply the elements of $\mathbb{Z}^*_{79}$ that are smaller than and coprime to its order 78:
      \begin{align*}
        \Big\langle \mathbb{Z}^*_{79}\Big\rangle = =\{&1, 5, 7, 11, 17, 19, 23, 25, \\
                                                      &29, 31, 35, 37, 41, 43, 47, 49,\\
                                                      &53, 55, 59, 61, 67, 71, 73, 77 \}\\
      \end{align*}

    \end{answer}
  \item Write the elements of $\mathbb{Z}^*_{79}$ as powers of 7.
    \begin{answer}
      Note all powers are evaluated considered $\bmod 79$ :\\~\\
      \begin{tabular}{|L|L|L|L|L|L|}
        \hline
        1=1^7 & 2=13^7 & 3=30^7 & 4=11^7 & 5=32^7 & 6=74^7\ \\
        \hline
        7=70^7 & 8=64^7 & 9=31^7 & 10=21^7 & 11=19^7 & 12=14^7 \\
        \hline
        13=16^7 & 14=41^7 & 15=12^7 & 16=42^7 & 17=27^7 & 18=8^7 \\
        \hline
        19=26^7 & 20=36^7 & 21=46^7 & 22=10^7 & 23=23^7 & 24=24^7 \\
        \hline
        25=76^7 & 26=50^7 & 27=61^7 & 28=59^7 & 29=39^7 & 30=77^7 \\
        \hline
        31=4^7 & 32=72^7 & 33=17^7 & 34=35^7 & 35=28^7 & 36=25^7 \\
        \hline
        37=34^7 & 38=22^7 & 39=6^7 & 40=73^7 & 41=57^7 & 42=45^7 \\
        \hline
        43=54^7 & 44=51^7 & 45=44^7 & 46=62^7 & 47=7^7 & 48=75^7 \\ 
        \hline
        49=2^7 & 50=40^7 & 51=20^7 & 52=18^7 & 53=29^7 & 54=3^7 \\
        \hline
        55=55^7 & 56=56^7 & 57=69^7 & 58=33^7 & 59=43^7 & 60=53^7  \\
        \hline
        61=71^7 & 62=52^7 & 63=37^7 & 64=67^7 & 65=38^7 & 66=63^7 \\
        \hline
        67=65^7 & 68=60^7 & 69=58^7 & 70=48^7 & 71=15^7 & 72=9^7 \\
        \hline
        73=5^7 & 74=47^7 & 75=68^7 & 76=49^7 & 77=66^7 & 78=78^7 \\
        \hline
      \end{tabular}
    \end{answer}
  \end{enumerate}

\newpage
\item Use Baby-step Giant-step Algorithm to compute following discrete logarithm problems:
  \begin{enumerate}
  \item $15 = 2^x \bmod 59$ (or equivalently $\log_2 15 \bmod 59$)
    \begin{answer}
      \begin{enumerate}[label=\arabic*.]
      \item Set up:
        \[\mathbb{Z}^*_{59}, t = \big\lfloor \sqrt{58} \big\rfloor = 7, q = 58\]
      \item Giant steps:
        \[k = 0 \ldots \big\lfloor \frac{q}{t} \big\rfloor = 8\]
        \begin{align*}
          &k = 0 & &g_0 = 2^{0\cdot 7} = 1 \bmod 59 \\
          &k = 1 & &g_1 = 2^{1\cdot 7} = 20 \bmod 59 \\
          &\ldots & & \\
          &k = 8 & &g_1 = 2^{8\cdot 7} = 15 \bmod 59  \text{ (found by brute-force accidentally)} \\
          &k = 9 & &g_1 = 2^{9\cdot 7} = 32 \bmod 59  \text{ (found by brute-force accidentally)} \\
        \end{align*}
        \[ g = \{1,20,56,29,54,9,15,\bm{32}\} \]
      \item Baby steps:
        \[i = 0 \ldots 7 , \alpha = 2:\]
        \begin{align*}
          &i = 0 & &h_0 = 15*2^{0} = 1 \bmod 59 \\
          &i = 1 & &h_1 = 15*2^{1} = 30 \bmod 59 \\
          &\ldots & & \\
        \end{align*}
        \[ h = \{1,30,1,2,4,8,16,\bm{32}\} \]
      \item Solve: 
        \begin{align*}
        h \cdot 2^7 &= h_7 = g_9 = 2^{63}\\
          2^7 \cdot h &= 2^{63}\\
          h &= 2^{56}\\
          x &= \log_2h = \bm{56}\\
        \end{align*}
      \end{enumerate}
    \end{answer}
    \newpage
  \item $23 = 11^x \bmod 79$ (or equivalently $\log_{11} 23 \bmod 79$)
    \begin{answer}
      \begin{enumerate}[label=\arabic*.]
      \item Set up:
        \[\mathbb{Z}^*_{79}, t = \big\lfloor \sqrt{78} \big\rfloor = 8, q = 78\]
      \item Giant steps:
        \[k = 0 \ldots \big\lfloor \frac{q}{t} \big\rfloor = 9\]
        \begin{align*}
          &k = 0 & &g_0 = 11^{0\cdot 8} = 1 \bmod 79 \\
          &k = 1 & &g_1 = 11^{1\cdot 8} = 44 \bmod 79 \\
          & \ldots & & \\
        \end{align*}
        \[ g = \{1,44,40,22,20,11,10,\bm{45},5,62\} \]
      \item Baby steps:
        \[i = 0 \ldots 8 , \alpha = 11\]
        \begin{align*}
          &i = 0 & &h_0 = 23*11^{0} = 1 \bmod 79 \\
          &i = 1 & &h_1 = 23*11^{1} = 16 \bmod 79 \\
          &\ldots & & \\
        \end{align*}
        \[ h = \{1,16,18,40,\bm{45}\} \]
      \item Solve: 
        \begin{align*}
          h \cdot 11^4 &= h_4 = g_7 = 11^{56}\\
          11^4 \cdot h &= 11^{56} \\
          h &= 11^{52}\\
          x &= \log_{11}h = \bm{52}\\
        \end{align*}
      \end{enumerate}
    \end{answer}
    \newpage
  \item $7 = 11^x \bmod 79$ (or equivalently $\log_{11} 7 \bmod 79$)
    \begin{answer}
      \begin{enumerate}[label=\arabic*.]
      \item Set up:
        \[\mathbb{Z}^*_{79}, t = \big\lfloor \sqrt{78} \big\rfloor = 8, q = 78\]
      \item Giant steps:
        \[k = 0 \ldots \big\lfloor \frac{q}{t} \big\rfloor = 9\]
        \begin{align*}
          &k = 0 & &g_0 = 11^{0\cdot 8} = 1 \bmod 79 \\
          &k = 1 & &g_1 = 11^{1\cdot 8} = 44 \bmod 79 \\
          & \ldots & & \\
        \end{align*}
        \[ g = \{1,44,40,22,20,11,10,45,5,62\} \]
      \item Baby steps:
        \[i = 0 \ldots 8 , \alpha = 11\]
        \begin{align*}
          &i = 0 & &h_0 = 7*11^{0} = 1 \bmod 79 \\
          &i = 1 & &h_1 = 7*11^{1} = 77 \bmod 79 \\
          &\ldots & & \\
        \end{align*}
        \[ h = \{1,77,57,74,24,27,60,28,31\} \]
      \item Cannot be solved: No such value $x$ exists
      \end{enumerate}
    \end{answer}
\newpage
  \item $100 = 7^x \bmod 103$ (or equivalently $\log_7 100 \bmod 103$)
    \begin{answer}
      \begin{enumerate}[label=\arabic*.]
      \item Set up:
        \[\mathbb{Z}^*_{103}, t = \big\lfloor \sqrt{102} \big\rfloor = 10, q = 102\]
      \item Giant steps:
        \[k = 0 \ldots \big\lfloor \frac{q}{t} \big\rfloor = 10\]
        \begin{align*}
          &k = 0 & &g_0 = 7^{0\cdot 10} = 1 \bmod 103 \\
          &k = 1 & &g_1 = 7^{1\cdot 10} = 15 \bmod 103 \\
          & \ldots & & \\
        \end{align*}
        \[ g = \{1,15,19,79,52,59,61,91,26,81,82\} \]
      \item Baby steps:
        \[i = 0 \ldots 10, \alpha = 7\]
        \begin{align*}
          &i = 0 & &h_0 = 100*7^{0} = 100 \bmod 103 \\
          &i = 1 & &h_1 = 100*7^{1} = 82 \bmod 103 \\
          &\ldots & & \\
        \end{align*}
        \[ h = \{1,82,\bm{59}\} \]
      \item Solve: 
        \begin{align*}
          h \cdot 7^2 &= h_2 = g_5 = 7^{50}\\
          7^2 \cdot h &= 7^{50} \\
          h &= 7^{48}\\
          x &= \log_{7}h = \bm{48}\\
        \end{align*}
      \end{enumerate}
    \end{answer}
    \newpage
  \item $100 = 7^x \bmod 101$ (or equivalently $\log_7 100 \bmod 101$)
    \begin{answer}
      \begin{enumerate}[label=\arabic*.]
      \item Set up:
        \[\mathbb{Z}^*_{101}, t = \big\lfloor \sqrt{100} \big\rfloor = 10, q = 100\]
      \item Giant steps:
        \[k = 0 \ldots \big\lfloor \frac{q}{t} \big\rfloor = 10\]
        \begin{align*}
          &k = 0 & &g_0 = 7^{0\cdot 10} = 1 \bmod 101 \\
          &k = 1 & &g_1 = 7^{1\cdot 10} = 65 \bmod 103 \\
          & \ldots & & \\
        \end{align*}
        \[ g = \{1,65,84,6,87,100,36,17,95,14,1\} \]
      \item Baby steps:
        \[i = 0 \ldots 10, \alpha = 7\]
        \begin{align*}
          &i = 0 & &h_0 = 100*7^{0} = 100 \bmod 101 \\
          &i = 1 & &h_1 = 100*7^{1} = 94 \bmod 101 \\
          &\ldots & & \\
        \end{align*}
        \[ h = \{1,94,52,61,23,60,16,11,77,34,\bm{36}\} \]
      \item Solve: 
        \begin{align*}
          h \cdot 7^{10} &= h_{10} = g_6 = 7^{60}\\
          7^{10} \cdot h &= 7^{60} \\
          h &= 7^{50}\\
          x &= \log_{7}h = \bm{50}\\
        \end{align*}
      \end{enumerate}
    \end{answer}
  \end{enumerate}

\newpage
\item D-H Key Exchange: Alice and Bob want to generate a common key. They agreed to use prime number $p = 809$ and generator $\alpha = 3$. Alice's private key = 17, Bob's private key = 41. Find the followings and show every intermediate step:
\begin{enumerate}
\item Alice's public key 
  \begin{answer}
    Alice and Bob's private keys 17 and 41 can be shown (\href{https://www.mathcelebrity.com/primitiveroot.php?p=809&b=17&pl=Primitive+Root+Check}{Alice}, \href{https://www.mathcelebrity.com/primitiveroot.php?p=809&b=41&pl=Primitive+Root+Check}{Bob}) to both be primitive elements of $\mathbb{Z}^*_{809}$. Therefore Alice's public key is simply
    \[ x = \alpha^{17} \bmod 809 = 302\]
  \end{answer}
\item Bob's public key 
  \begin{answer}
    And Bob's public key is simply
    \[ y = \alpha^{41} \bmod 809 = 153 \]
  \end{answer}
\item Common key generated by Alice and Bob
  \begin{answer}
    Since the keys are primitive roots, the following 
    \[ x^{17} \bmod 809 = y^{41} \bmod 809 = 410\]
  \end{answer}
\end{enumerate}

\end{enumerate}


\end{document}