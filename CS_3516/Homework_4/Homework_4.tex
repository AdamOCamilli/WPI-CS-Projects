\documentclass[12pt]{article}

\usepackage{verbatim}
\usepackage[left=1in, right=1in, top=1in, bottom=1in]{geometry}
\usepackage{amsmath}

\begin{document}

\title{Homework 4}
\author{Adam Camilli}
\date{\today}
\maketitle

\begin{enumerate}
\item If a binary signal is sent over a 3 kHz channel whose signal-to-noise ratio is 20 dB, waht is the maximum achievable data rate?
  \\ \\
  We are given:
  \begin{itemize}
  \item Number of levels in \textit{binary} signal $\Rightarrow$ 2
  \item The signal-to-noise ratio $\Rightarrow$ 20 dB
  \end{itemize}
  So we need to consider the maximum data rate according to Shannon's theorem, and the Nyquist limit for binary signalling over a 3 kHz channel.
\\ \\
First, Shannon's Theorem gives:
\[ \textrm{Max Data Rate} = Blog_2(1 + \frac{S}{N}) \]
Since $\frac{S}{N}$ is a power ratio, we must convert from decibels to this power ratio:
\[ \textrm{Signal-Noise Ratio in dB} = 10\log_{10}(\frac{S}{N}) \]
\[ \frac{S}{N} = 10^{\frac{20}{10}} = 100 \]
Therefore, according to Shannon's theorem, the maximum data rate is: 
\[ \textrm{Max Data Rate} = 3000 \cdot \log_2(1 + 100) \approx 20 \textrm{ kbps} \]
The Nyquist limit for binary signalling over a 3kHz channel, however, is only 
\[ \textrm{Max Data Rate} = 2\cdot B\log_{2}(V\textrm{ levels}) = 2 \cdot 3000 \cdot \log_{2}2 = 6 \textrm{ kbps} \]
Therefore the maximum \textit{achievable} data rate for a 3 kHz channel is \fbox{6 kbps}.

\newpage

\item How much bandwidth is there in 0.1 micron of spectrum at a wavelength of 1 micron?
\\ \\
Since we are dealing with a spectrum for a fiber optic cable, bandwidth is given by
\[ B = \frac{c \cdot \delta_{\lambda}}{\lambda} \]
where $c = 300000000$ m/s, or the speed of light, $\lambda = 1$ micron, and $\delta_{\lambda} = 0.1$ microns. We therefore have approximately \fbox{$\frac{3 \cdot 10^{13}}{1^2}$ Hz $ = 30$ THz} of bandwidth. 

\item Radio antennas often work best when the diameter of the antenna is equal to the wavelength of the radio wave. Reasonable antennas range from 1 cm to 5 meters in diameter. What frequency range does this cover?
\\ \\
The frequency of any wave is given by the velocity (the speed of light here) over the wavelength: 
\[ f = \frac{c}{\lambda} \]
Thus our range will be the frequency for antennas 5 m in length to the frequency for antennas 1 cm in length:
\[ f_l = \frac{3 \cdot 10^8}{5} = 30 \textrm{GHz} \]
\[ f_h = \frac{3 \cdot 10^8}{0.01} = 60 \textrm{MHz} \]
This covers the frequency range of \fbox{60 MHz to 30 GHz}.

\item Ten signals, each requiring 4000 Hz, are multiplexed on to a single channel using FDM. How much minimum band-
width is required for the multiplexed channel? Assume that the guard bands are 400 Hz wide. 
\\ \\
To avoid interference, we need a guard band to separate the upper and lower edges of the 10 signals, meaning 11 guard bands are needed total. 
\[ \boxed{4000 \cdot 10 + 400 \cdot 11 = \textrm{44400 Hz}} \]

\newpage

\item Suppose that x bits of user data are to be transmitted over a k-hop path in a packet-switched network as a series of  packets, each containing p data bits and h header bits. Assume x>>p+h. The line speed is b bits per sec and the propagation delay is negligible.  What value of p minimizes the total delay? 
\\ \\
The total number of packets necessary for the transmission is $\frac{x}{p}$. \\
The total bits to transter including data and the headers which are not included in data is $\frac{(p+h)\cdot x}{p}$ bits. \\
To transmit all these bits takes $\frac{(p+h)\cdot x}{bp}$ sec. \\ 
The last packet takes more time to travel by intermediate routing: $\frac{(k-1)(p+h)}{b}$ sec. The total time is thus: 
\[ \textrm{Total Time} = \frac{(p+h)\cdot x}{bp} + \frac{(k-1)(p+h)}{b} \]

To minimize this, we differentiate with respect to p, then set it equal to zero and solve for p: \[ \frac{(k-1)p^2-hx}{bp^2} = 0 \]
\[ \boxed{ p = \sqrt{\frac{hx}{k-1}} } \]



\end{enumerate}

\end{document}