\message{ !name(Homework_1.tex)}\documentclass[12pt]{article}
\begin{document}

\message{ !name(Homework_1.tex) !offset(-3) }

Adam Camilli \\
Homework 1 \\
CS 3516\\
May 25, 2017\\

\begin{enumerate}
\item Imagine that you have trained your St. Bernard, Bernie, to carry a box of three 8mm tapes instead of a flask of brandy. (When your disk fills up, you consider that an emergency). These tapes each contain 7 gigabytes. The dog can travel to your side, wherever you may be, at 18 km/hour. For what range of distances does Bernie have a higher data rate than a transmission line whose data rate (excluding overhead) is 150 Mbps? \\

  The dog can carry 21 gigabytes, or 168 gigabits. A speed of 18 km/hour = 0.005 km/sec. The time to travel distance x km is x/0.005 = 200x sec, leading toa data rate of (168/200)x Gbps or 840/x Mbps. \\
  $840/x > 150$ \\
  $840 > 150x$ \\
  $x < 28/5$ \\
  For $x < 5.6 km$, the dog has a higher data rate than the transmisssion line.
  \begin{enumerate}
  \item If Bernie's speed is doubled, his speed of 36 km/hour = 0.01 km/sec. The time to travel distance x km is x / 0.01 = 100x sec, leading to a data rate of (168/100x) Gbps or 1680/x Mbps. \\
    1680/x > 150 \\
    1680 > 150x \\
    x < 56/5 \\
    For x < 11.2 km, the dog has a higher data rate than the transmission line.
  \end{enumerate}
\end{enumerate}

\end{document}
\message{ !name(Homework_1.tex) !offset(-29) }
