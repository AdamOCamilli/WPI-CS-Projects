\documentclass{article}
\usepackage[utf8]{inputenc}
\usepackage[english]{babel}
\usepackage[]{amsthm} %lets us use \begin{proof}
\usepackage[]{amssymb} %gives us the character \varnothing
\usepackage{graphicx}

\title{Homework 1}
%\author{Your Name}
\date{}
%This information doesn't actually show up on your document unless you use the maketitle command below

\begin{document}
\maketitle %This command prints the title based on information entered above

%Section and subsection automatically number unless you put the asterisk next to them.
\section*{Problem 1}
Find the least integer k such that $f(n)$ is $O(n^k)$ for each of the following
functions. Include values for $c$ and $n_0$ as described in section 3.1, page 47 of the textbook.
\begin{itemize}
\item$f(n) = 2n^2 + n^3\log n$
\item$f(n) = 3n^5 + (\log n)^4$
\item$f(n) = (n^4 + n^2 + 1)/(n^4 + 1)$
\item$f(n) = (n^3 + 5 \log n)/(n^4 + 1)$
\end{itemize}

%\subsection{solution}

\section*{Problem 2}
You have $n$ quarters and a balance. You know that $n-1$ quarters have
the same weight, and one weighs less than the others. Give an algorithm
(in pseudocode) to identify the light quarter which uses the balance only
$\log_{3}n$ times in the worst case.

%\subsection{solution}

\section*{Problem 3}
Use the Master Theorem to find the asymptotic solutions for the following recurrences

\begin{itemize}
\item $T(n) = 7T(\frac{n}{2}) + n^2$
\item $T(n) = T (\frac{n}{2}) + 1$
\item $T(n) = 4T(\frac{n}{2}) + n^3$
\end{itemize}

\section*{Problem 4}
$A[1...n]$ is a \textbf{sorted} array of \textbf{distinct} integers. We want to decide whether there is an index $i$ where $A[i] = i$.

\begin{itemize}
\item Describe a divide-and-conquer algorithm that solves this problem.
\item Use the Master Theorem to estimate the running time of the algorithm. Your algorithm should run in $O(\log n)$ time
\end{itemize}

%\subsection{solution}

\section*{Problem 5}
Suppose you are tossing $m$ balls into $n$ bins. Each ball is equally likely to land in each bin, and the ball tosses are independent. What is the expected number of bins that contain exactly $k$ balls? Use indicator random variables to find the solution.

%\subsection{solution}

\section*{Problem 6}
Use a recursion tree to determine a good asymptotic upper bound on the recurrence
$T(n) = T(\frac{n}{2}) + n^2$. Use the substitution method to verify your answer.

%\subsection{solution}

\section*{Problem 7}
Using Figure 1 as a model (also can be find in the textbook page 161), illustrate the operation of HEAPSORT on the array
$A=[5, 13, 12, 25, 71, 37, 27, 9, 22]$.
%\input{fig_heapsort}

\section*{Problem 8}
Using QUICKSORT to sort the array $A=[5, 13, 12, 25, 71, 37]$. You just need to show the result after each round. Here is a example for $A=[2, 8, 7, 1, 3, 5, 6, 4]$, suppose you pick the last element in a region as its pivot:

round 1: region=$A$, result=$[2, 1, 3, 4, 7, 5, 6, 8]$

round 2: region$_1$=$[2, 1, 3]$, region$_2$=$[7, 5, 6, 8]$, result=$[2, 1, 3, 4, 7, 5, 6, 8]$. 

round 3: region$_1$=$[2,1]$, region$_2$=$[7, 5, 6]$, result=$[1, 2, 3, 4, 5, 6, 7, 8]$.

The fig 2 shows the the detail operations during the round 1.

%\input{fig_quicksort}

\section*{Problem 9}
The input is two sets $S1$ and $S2$ containing $n$ real numbers in total, and a real number $x$.

(a) Find a O($n$log$n$) time algorithm that determines whether there exists an element from $S1$ and an element from $S2$ whose sum is exactly $x$.

(b) Suppose now that the two sets are given in sorted order. Find a O($n$)-time algorithm solving this problem.

You can either show pseudo code or describe it in English.

\section*{Problem 10}
Show that $2n-1$ comparisons are necessary in the worst case to merge
two sorted lists containing $n$ elements each.


%If you want centered math on its own line, you can use a slash and square bracket.\\
%\[
%\left \{
%\sum\limits_{k=1}^\infty l(I_k):A\subseteq \bigcup_{k=1}^\infty \{I_k\}
%\right \}
%\]
%The left and right commands make the brackets get as big as we need them to be.
%
%\clearpage %Gives us a page break before the next section. Optional.

\section*{Problem 11 $\star$}
Show an example that COUNTING SORT can be slower than any comparison sorts you have learned. (This problem is for practice,  but will not be counted towards the grade)

\end{document}